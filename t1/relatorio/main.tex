\documentclass[12pt]{article}
\usepackage[brazil]{babel}
\usepackage{graphicx}
\usepackage{amsmath}
\usepackage{caption}
\usepackage{graphicx}
\usepackage{geometry}
\usepackage{hyperref}
\usepackage{minted}

\hypersetup{
    colorlinks=true,
    linkcolor=blue,
    linktoc=all
}

\usemintedstyle{vs} % Try 'friendly', 'vs', 'autumn', etc.

\geometry{top=3cm, bottom=2cm, left=3cm, right=2cm} % Margins

\begin{document}

\tableofcontents % Gera o índice automaticamente
\newpage

\section{Introdução}
    O objetivo deste trabalho é realizar alguns processamentos básicos em imagens digitais.


\section{Estrutura do trabalho}
    \subsection{Requisitos}
    A versão utilizada do Python foi a 3.12.3, e as versões das bibliotecas utilizadas no projeto são as que seguem:
\begin{itemize}
    \item NumPy: 2.2.4
    \item Matplotlib: 3.10.1
    \item OpenCV: 4.11.0.86
\end{itemize}
    \subsection{Organização dos arquivos}
    \input{organizacao-arquivos.tex}
    \subsection{Funções auxiliares}
    No arquivo \texttt{helper\_functions.py} foram implementadas algumas funções que foram utilizadas, principalmente, para análise dos exercícios.

\begin{itemize}
    \item \texttt{display\_image\_grid}: Essa função exibe uma grade de imagens, permitindo visualizar várias imagens de uma só vez. Ela recebe como parâmetros um dicionário de imagens e o número de linhas e colunas desejadas na grade. A função organiza as imagens em uma grade e exibe cada uma delas com seu respectivo título; contudo, para que a grade caiba em uma tela de $1600$x$900$, a função redimensiona as imagens caso necessário, mantendo as mesmas proporções entre comprimento e largura.
\end{itemize}


\section{Exercícios}
    \subsection{Esboço a Lápis}
    \definecolor{LightGray}{gray}{0.9}
\begin{minted}[
    frame=lines,
    framesep=2mm,
    baselinestretch=1.2,
    bgcolor=LightGray,
    fontsize=\footnotesize,
    linenos
]{python}
    import cv2
    img = cv2.imread("image.png")
    cv2.imshow("Display", img)
\end{minted}


    \subsection{Ajuste de Brilho}
    \input{ex02-ajuste-brilho.tex}
    \subsection{Mosaico}
    \input{ex03_mosaico.tex}
    \subsection{Alteração de Cores}
    \input{ex04_alteracao_cores.tex}
    \subsection{Transformação de Imagens Coloridas}
    \input{ex05_transformacao_imgs_coloridas.tex}
    \subsection{Planos de Bits}
    \input{ex06_plano_bits.tex}
    \subsection{Combinação de Imagens}
    \input{ex07_combinacao_imagens.tex}
    \subsection{Transformação de Intensidade}
    \input{ex08_transformacao_intensidade.tex}
    \subsection{Quantização de Imagens}
    \input{ex09_quantizacao_imagens.tex}
    \subsection{Filtragem de Imagens}
    \input{ex10_filtragem_imagens.tex}

\section{Referências}
    \begin{thebibliography}{99}

\bibitem{example1}
ACEROLA, This is the Difference of Gaussians. YouTube, 24 de dezembro de 2022. Disponível em: https://www.youtube.com/watch?v=5EuYKEvugLU. Acesso em: 29 de março de 2025.

\end{thebibliography}

% Outras seções podem ser adicionadas aqui

\end{document}
